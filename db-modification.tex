\documentclass[12pt]{article}
\usepackage{graphicx}
\usepackage{float}
\usepackage{amsmath}
\usepackage{sidecap}
\usepackage{fullpage}
\usepackage{hyperref}
\usepackage{listings}
\usepackage{latexsym}
\usepackage{color}
\usepackage{tikz}
\usetikzlibrary{shapes,arrows, matrix, positioning, fit}

% Java code with lstlisting
\definecolor{dkgreen}{rgb}{0,0.6,0}
\definecolor{gray}{rgb}{0.5,0.5,0.5}
\definecolor{mauve}{rgb}{0.58,0,0.82}

\lstset{frame=tb,
    language=Java,
    aboveskip=3mm,
    belowskip=3mm,
    showstringspaces=false,
    columns=flexible,
    basicstyle={\small\ttfamily},
    numbers=none,
    numberstyle=\tiny\color{gray},
    keywordstyle=\color{blue},
    commentstyle=\color{dkgreen},
    stringstyle=\color{mauve},
    breaklines=true,
    breakatwhitespace=true
    tabsize=3
}

\title{CIS4301 Notes: }
\author{Ryan Roden-Corrent}
\date{\today}

\begin{document}
\setlength\parindent{0pt}
% Tikz general settings
\tikzstyle{relation} = [diamond, draw, fill=blue!20, text width=4em,
  text badly centered, node distance=3cm, inner sep=0pt]
\tikzstyle{attribute} = [draw, ellipse, fill=red!20, node distance=2.5cm,
  minimum height=2em]
\tikzstyle{entity} = [rectangle, draw, fill=blue!20, text width=5em,
  text centered, minimum height=4em]
\tikzstyle{relation-weak} = [diamond, double, draw, fill=blue!20, text width=4em,
  text badly centered, node distance=3cm, inner sep=0pt]
\tikzstyle{entity-weak} = [rectangle, draw, double, fill=blue!20, text width=5em,
  text centered, minimum height=4em]
\tikzstyle{line} = [draw, -]
\tikzstyle{arrow} = [draw, -latex', thick]
\tikzstyle{arrow-round} = [draw, -), thick]
\maketitle

\section{Database Modifications}
\subsection{Insert}
\begin{lstlisting}[language=sql,caption=multiple value insertion]
  INSERT INTO Likes
  VALUES ('Sally', 'Bud'), ('Jim','Miller');  --comma separated tuples to enter
\end{lstlisting}
\subsubsection{Default Values}
\begin{lstlisting}[language=sql,caption=price defaults to 5 if not specified]
  ...,
  price Real DEFAULT 5,   --make sure price is a reasonable, non-NULL value
  ...,
\end{lstlisting}
\begin{lstlisting}[language=sql,caption=another default example]
  CREATE TABLE Drinkers (
    name CHAR(30) PRIMARY KEY,
    addr CHAR(50)
      DEFAULT '123 Sesame St.',
    phone CHAR(16)
  );
\end{lstlisting}
\subsubsection{Subqueries in insertion}
\begin{lstlisting}[language=sql,caption=insertion via subquery]
  INSERT INTO PotBuddies
  (
    SELECT d2.drinker
    FROM Frequents d1, Frequents d2
    WHERE d1.drinker = 'Sally AND
    d2.drinker <> 'Sally' AND
    d1.bar = d2.bar
  );
\end{lstlisting}
Find all the drinkers at the bars Sally frequents and insert them into
PotBuddies (Potential Buddies, what did you think it stands for?)\\
\begin{tabular}{|c|c|}
  \hline
  d1.bar & d2.bar\\
  \hline
  'Sally' & NOT 'Sally'\\
  'Sally' & NOT 'Sally'\\
  \hline
\end{tabular}

\subsection{Deletion}
\begin{lstlisting}[language=sql, caption=Sally no longer likes Bud]
  DELETE FROM Likes
  WHERE drinker = 'Sally' AND
    beer = 'Bud';
\end{lstlisting}
Delete all rows where the drinker is 'Sally' and the beer is 'Bud'
\begin{lstlisting}[language=sql,caption=clear out entire table]
  DELETE FROM Likes;  -- no WHERE clause needed
\end{lstlisting}

\begin{lstlisting}[language=sql,caption=delete with subquery]
  DELETE FROM Beers b
  WHERE EXISTS (  --check if another beer is made by the same manufacturer
    SELECT name FROM Beers  --implicit join of Beers with itself
    WHERE manf = b.manf AND
      name <> b.name
  );
\end{lstlisting}
Delete all beers where there is another beer by the same manufacturer.\\
\begin{tabular}{|c|c|c}
  \hline
  name & manf&\\
  \hline
  Bud & Budweiser & mark as dirty\\
  BudLite & Budweiser & mark as dirty\\
  \hline
\end{tabular}\\
Delete is a \textbf{mark-and-sweep} process: first mark items for deletion, then
delete all marked items. (If items were deleted immediately, it could disrupt
the condition for deleting other items during the same deletion process).

\subsection{Updates}
\begin{lstlisting}[language=sql,caption=UPDATE template]
  UPDATE <relation>
  SET <list of attribute assignments>
  WHERE <condition on tuples>;
\end{lstlisting}

\begin{lstlisting}[language=sql,caption=Change Fred's Phone number]
  UPDATE Drinkers
  SET phone = '555-1212'
  WHERE name = 'Fred';
\end{lstlisting}

\begin{lstlisting}[language=sql,caption=set maximum price on beers]
  UPDATE Sells
  SET price = 4.00
  WHERE price > 4.00;
\end{lstlisting}

\begin{lstlisting}[language=sql,caption=add tax to price]
  UPDATE Sells
  SET price = 1.05 * price  --value can be result of a computation on attributes
  WHERE price > 4.00;
\end{lstlisting}

\section{Constraints}
\begin{description}
    \item[constraint] relations enforced by DBMS
    \item[trigger] only executed when a condition occurs
\end{description}

\begin{description}
    \item[Keys] 
    \item[Foreign-keys] referential integrity
    \item[value-based] constrain value of attribute
    \item[tuple-based] relationships between components
    \item[assertions] boolean expression
\end{description}

\subsection{Keys}
\subsubsection{Single Attribute Keys}
\begin{lstlisting}[language=sql,caption=ensure names are unique]
  CREATE TABLE Beers (
    name CHAR(20) UNIQUE,   --note: name can still be NULL!
    manf CHAR(20)
  );
\end{lstlisting}

\subsubsection{Multi Attribute Keys}
\begin{lstlisting}[language=sql,caption=tuple as a primary key]
  CREATE TABLE Sells (
  bar CHAR(20),
  beer VARCHAR(20),
  price REAL,
  PRIMARY KEY (bar,beer));
\end{lstlisting}

\subsubsection{Foreign Keys}
\subsection{Foreign Keys}
Indicate that a key REFERENCES another relation and is used as a key.\\
Referenced attributes must be declared PRIMARY KEY or UNIQUE.
\begin{lstlisting}[language=sql,caption=tuple as a primary key]
  CREATE TABLE Sells (
  bar CHAR(20),
  beer VARCHAR(20),
  price REAL,
  FOREIGN KEY (beer) REFERENCES Beer);
\end{lstlisting}

\end{document}

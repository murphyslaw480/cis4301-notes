\documentclass{article}
\usepackage{graphicx}
\usepackage{float}
\usepackage{qtree}
\usepackage{sidecap}
\usepackage{fullpage}
\usepackage{hyperref}
\usepackage{listings}
\usepackage{color}
\usepackage{tikz}
\usetikzlibrary{shapes,arrows, matrix, positioning, fit}
\DeclareGraphicsExtensions{.pdf,.png,.jpg}
\graphicspath{ {img/} }

% Java code with lstlisting
\definecolor{dkgreen}{rgb}{0,0.6,0}
\definecolor{gray}{rgb}{0.5,0.5,0.5}
\definecolor{mauve}{rgb}{0.58,0,0.82}

\lstset{frame=tb,
    language=Java,
    aboveskip=3mm,
    belowskip=3mm,
    showstringspaces=false,
    columns=flexible,
    basicstyle={\small\ttfamily},
    numbers=none,
    numberstyle=\tiny\color{gray},
    keywordstyle=\color{blue},
    commentstyle=\color{dkgreen},
    stringstyle=\color{mauve},
    breaklines=true,
    breakatwhitespace=true
    tabsize=3
}

\title{CIS4301 Notes}
\author{Ryan Roden-Corrent}
\date{Wed Jan 22 10:49:19 EST 2014}

\begin{document}
\setlength\parindent{0pt}
% Tikz general settings
\tikzstyle{relation} = [diamond, draw, fill=blue!20, text width=4em,
  text badly centered, node distance=3cm, inner sep=0pt]
\tikzstyle{attribute} = [draw, ellipse, fill=red!20, node distance=2.5cm,
  minimum height=2em]
\tikzstyle{entity} = [rectangle, draw, fill=blue!20, text width=5em,
  text centered, minimum height=4em]
\tikzstyle{relation-weak} = [diamond, double, draw, fill=blue!20, text width=4em,
  text badly centered, node distance=3cm, inner sep=0pt]
\tikzstyle{entity-weak} = [rectangle, draw, double, fill=blue!20, text width=5em,
  text centered, minimum height=4em]
\tikzstyle{line} = [draw, -]
\tikzstyle{arrow} = [draw, -latex', thick]
\tikzstyle{arrow-round} = [draw, -), thick]
\maketitle

\section{Entity Relationship (ER) Models}
\subsection{Example ER Model}

\begin{figure}[H]
\begin{tikzpicture}[node distance = 2cm, auto]
  \node [entity-weak] (logins) {logins};
  \node [attribute, left of=logins] (billing) {billing};
  \node [attribute, above of=logins] (name) {\underline{name}};
  \node [relation-weak, right of=logins] (at) {at};
  \node [entity, right of=at, node distance=3cm] (host) {host};
  \node [attribute, above of=host] (hostname) {\underline{name}};
  \node [attribute, right of=host] (url) {url};

  \path [line] (billing) -- (logins);
  \path [line] (name) -- (logins);

  \path [line] (logins) -- (at);
  \path [arrow] (at) -- (host);

  \path [line] (hostname) -- (host);
  \path [line] (url) -- (host);
\end{tikzpicture}
\end{figure}

Host is a \textbf{strong entity} (can exist on its own)\\
Login is a \textbf{weak entity} (cannot exist without Host)\\
\begin{description}
  \item[Host] uniquely identified by name
  \item[name] underlined - key for Host (must be unique)
  \item[URL]
  \item[At] Circular arrow - maps to a single Host, cannot exist without it
  \item[Logins] A login requires a Host
  \item[name] underlined - key for name (must be unique)
  \item[Host]
\end{description}

\subsection{Schema Format}
\begin{itemize}
  \item{Host(\underline{name}, URL)}
  \item{Logins(\underline{name}, \underline{host.name}), billing}
\end{itemize}

\pagebreak
\section{ER Diagrams}
\begin{figure}[h!]
  \centering
  \caption{Example ER Diagram}
\begin{tikzpicture}[node distance = 2cm, auto]
  \node [entity] (stars) {stars};
  \node [attribute, above of=stars] (name) {\underline{name}};
  \node [attribute, above left of=stars] (DoB) {\underline{DoB}};
  \node [attribute, left of=stars] (other) {\ldots};
  \node [relation, right of=stars] (contracts) {contracts};
  \node [attribute, above of=contracts] (salary) {salary};
  \node [entity, right =1.5cm of contracts] (movies) {movies};
  \node [attribute, above of=movies] (title) {\underline{title}};
  \node [entity, below of=contracts] (studios) {studios};
  \node [attribute, left of=studios] (room) {\underline{room}};

  \path [line] (stars) -- (name);
  \path [line] (stars) -- (DoB);
  \path [line] (stars) -- (other);
  \path [line] (stars) -- (contracts);
  \path [line] (contracts) -- (movies);
  \path [line] (contracts) -- (salary);
  \path [line] (title) -- (movies);
  \path [line] (studios) -- (room);

  \path [arrow] (contracts) -- (studios);

\end{tikzpicture}
\end{figure}
\begin{description}
    \item[stars-movies]{many-many}
    \item[stars-studios]{many-one}
    \item[movies-studios]{many-one}
\end{description}
An \underline{underlined attribute} indicates a key for the entity. \\
A star can only be associated with one studio, but a studio can be associated
with many stars.\\
A movie can only be associated with one studio, but a studio can be associated
with many movies.\\\\
\textbf{Schema:}
\begin{itemize}
    \item{stars(\underline{name}, \underline{DoB}, \ldots)}
    \item{stars(\underline{title}, )}
    \item{Contracts(\underline{StudioName}, \underline{starname},
        \underline{DoB}, \underline{movie-title}, salary)}
\end{itemize}
%MAKE THIS A TABLE
%
Example Contracts:\\
\begin{itemize}
    \item{(Ricky R, Jaws, MGM)}
    \item{(Ricky R, Batman, WB)}
    \item{(Michael Keaton, Batman, WB)}
\end{itemize}
StudioName, StarName, and MovieTitle are all required to uniquely identify a
contract.
\pagebreak

\subsection{Multiple Key Example}
If there are no underlined attributes for an entity, the entire collection of
attributes is the unique identifier for that entity. Anything short of that is
not sufficient to uniquely identify the entity.

\begin{figure}[H]
  \centering
  \caption{A car is uniquely identified by the tuple (make,model,color,year)}
  \begin{tikzpicture}[node distance = 2cm, auto]
    \node [entity] (car) {car};
    \node [attribute, above of=car] (make) {make};
    \node [attribute, above left of=car] (model) {model};
    \node [attribute, above right of=car] (color) {color};
    \node [attribute, right of=car] (year) {year};
    \path [line] (car) -- (make);
    \path [line] (car) -- (model);
    \path [line] (car) -- (color);
    \path [line] (car) -- (year);
  \end{tikzpicture}
\end{figure}
\subsection{Weak Entity Example}
\begin{figure}[H]
  \centering
  \caption{A car cannot exist without a VIN issued by a Dealer}
  \begin{tikzpicture}[node distance = 2cm, auto]
    \node [entity-weak] (car) {car};
    \node [attribute, above of=car] (make) {make};
    \node [attribute, above left of=car] (model) {model};
    \node [attribute, above right of=car] (color) {color};
    \node [attribute, right of=car] (year) {year};
    \node [relation-weak, left=2cm of car] (issues) {Issues};
    \node [entity, left=2cm of issues] (dealer) {dealer};
    \node [attribute, above of=issues] (VIN) {VIN};
    \path [line] (car) -- (make);
    \path [line] (car) -- (model);
    \path [line] (car) -- (color);
    \path [line] (car) -- (year);
    \path [line] (car) -- (issues);
    \path [line] (issues) -- (VIN);
    \path [arrow] (issues) -- (dealer);
  \end{tikzpicture}
\end{figure}
Issues(\underline{VIN},\underline{Dealer})

\subsection{Weak Entities and Referential Integrity}
The difference between weak entities and referential integrity (the rounded
arrowhead) can be subtle and confusing, so here's the deal:\\

\begin{figure}[H]
  \centering
  \caption{No Referential Integrity}
  \label{weak-ref1}
\begin{tikzpicture}[node distance = 2cm, auto]
  \node [entity] (movies) {movies};
  \node [relation, right of=movies] (owns) {owns};
  \node [attribute, left of=movies] (movie-name) {\underline{name}};
  \node [entity, right = 1cm of owns] (studio) {studio};
  \node [attribute, right of=studio] (studio-name) {\underline{name}};

  \path [line] (movies) -- (movie-name);
  \path [line] (movies) -- (owns);
  \path [arrow] (owns) -- (studio);
  \path [line] (studio) -- (studio-name);
\end{tikzpicture}
\end{figure}
In Figure~\ref{weak-ref1}, the arrow indicates a \textbf{many-one} relation
from movies to studio. This means that a movie may be owned by at most one
studio. This allows for the possibility that a movie is \emph{not owned by any
  studio}. If we want to enforce that a movie be owned by a studio, we must use
a rounded arrowhead to represent \textbf{Referential Integrity}.

\begin{figure}[H]
  \centering
  \caption{Referential Integrity}
  \label{weak-ref2}
\begin{tikzpicture}[node distance = 2cm, auto]
  \node [entity] (movies) {movies};
  \node [relation, right of=movies] (owns) {owns};
  \node [entity, right = 1cm of owns] (studio) {studio};
  \node [attribute, left of=movies] (movie-name) {\underline{name}};
  \node [attribute, right of=studio] (studio-name) {\underline{name}};

  \path [line] (movies) -- (owns);
  \path [arrow-round] (owns) -- (studio);
  \path [line] (movies) -- (movie-name);
  \path [line] (studio) -- (studio-name);
\end{tikzpicture}
\end{figure}
In Figure~\ref{weak-ref2}, the rounded arrow indicates a \textbf{many-one}
relation with \textbf{Referential Integrity}. This means that for every movie
entity, there must be a studio entity that it is associated with. Note that,
even in this example, \textbf{movies is not a weak entity}. Why? Because even
though a movie must be associated with a studio, locating a particular movie
does not require any information about the producing studio. A movie can be
uniquely identified with just its \underline{name}.\\
What follows is a truly weak entity, based on an example a classmate posted on
piazza:

\begin{figure}[H]
  \centering
  \caption{Room: A Weak Entity}
  \label{weak-ref3}
\begin{tikzpicture}[node distance = 2cm, auto]
  \node [entity-weak] (room) {room};
  \node [relation-weak, right of=movies] (inside) {inside};
  \node [entity, right = 1cm of owns] (building) {building};
  \node [attribute, left of=movies] (number) {\underline{number}};
  \node [attribute, right of=studio] (name) {\underline{name}};

  \path [line] (room) -- (inside);
  \path [arrow-round] (inside) -- (building);
  \path [line] (room) -- (number);
  \path [line] (building) -- (name);
\end{tikzpicture}
\end{figure}
In Figure~\ref{weak-ref3}, room is a weak entity. This means that a room's
number is not sufficient to uniquely identify it -- a building name is also
required. As an example, the number G186 does not uniquely identify a room, but
McCarty G186 does. Note that the referential integrity arrow is also necessary,
as a room must be associated with a building.

\section{Postgres}
See \url{http://www.postgresql.org/docs/9.2/static/index.html} for full
documentation.\\
See \url{http://sqlfiddle.com/} for practice.
\subsection{SQL Server Commands}
\begin{description}
    \item[ ssh username@thunder.cise.ufl.edu ]{ssh into CISE thunder server}
    \item[ psql -h postgres.cise.ufl.edu db ]{enter database db}
    \item[\textbackslash ?] { view commands}
    \item[\textbackslash l] {list databases}
    \item[\textbackslash d] {to list relations}
    \item[\textbackslash d] {to list relations}
    \item[TRUNCATE TABLE mytable] {remove all data from table mytablej}
    \item[DROP TABLE mytable] {completely remove table mytable}
    \item[ALTER TABLE mytable DROP COLUMN date] {remove column date from mytable}
\end{description}

\subsection{SQL Data Types}
\begin{samepage}
\begin{description}
    \item[CHAR, CHAR(1)] {fixed number of strings available (e.g. gender: M,F)}
    \item[VARCHAR(n)] {variable length strings up to specified size n}
    \item[BIT(n)] {bit string - fixed sized n}
    \item[BIT VARYING (n)] {Variable length bit string up to n bits}
    \item[BOOLEAN] {}
    \item[INT, INTEGER] {}
    \item[SHORT INT] {}
    \item[FLOAT,REAL] {single precision FP}
    \item[DOUBLE] {Double precision FP}
    \item[DECIMAL(n,d)] {fixed n digits above decimal, d below decimal}
    \item[DATE] {'YYYY-MM-DD', can be added as string following preceding format}
    \item[TIME] {'HH:MM:SS', seconds are in decimal (e.g. '15:00:02.5')}
    \item[DATETIME] {}
    \item[TIMESTAMP] {}
\end{description}
\end{samepage}

\subsection{Example Database}
\begin{figure}[H]
  \centering
  \caption{Example Database}
\begin{tikzpicture}[node distance = 2cm, auto]
  \node [entity] (weather) {weather};
  \node [attribute, left of=weather] (city) {city};
  \node [attribute, above left of=weather] (temp-high) {temp-high};
  \node [attribute, above of=weather] (temp-lo) {temp-lo};
  \node [attribute, above right of=weather] (prcp) {prcp};
  \node [attribute, right of=weather] (date) {date};

  \path [line] (weather) -- (city);
  \path [line] (weather) -- (prcp);
  \path [line] (weather) -- (date);
  \path [line] (weather) -- (temp-lo);
  \path [line] (weather) -- (temp-high);
\end{tikzpicture}
\end{figure}

\begin{lstlisting}[language=SQL]
CREATE TABLE weather (
  city    varchar(80) REFERENCES cities (name),
  temp_lo int DEFAULT 70,   --use defaults to avoid null
  temp_hi int,
  prcp    real,
  date    date,
  PRIMARY KEY (city, date),
);
\end{lstlisting}

\end{document}

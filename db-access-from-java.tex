\documentclass[12pt]{article}
\usepackage{graphicx}
\usepackage{sverb}
\usepackage{float}
\usepackage{amsmath}
\usepackage{sidecap}
\usepackage{fullpage}
\usepackage{hyperref}
\usepackage{listings}
\usepackage{latexsym}
\usepackage{color}
\usepackage{tikz}
\usetikzlibrary{shapes,arrows, matrix, positioning, fit}

\title{CIS4301 Notes: Frontend Design}
\author{Ryan Roden-Corrent}
\date{\today}

\definecolor{lightgray}{rgb}{.9,.9,.9}
\definecolor{darkgray}{rgb}{.4,.4,.4}
\definecolor{purple}{rgb}{0.65, 0.12, 0.82}

\lstdefinelanguage{JavaScript}{
  keywords={typeof, new, true, false, catch, function, return, null, catch,
    switch, var, if, in, while, do, else, case, break},
  keywordstyle=\color{blue}\bfseries,
  ndkeywords={class, export, boolean, throw, implements, import, this},
  ndkeywordstyle=\color{darkgray}\bfseries,
  identifierstyle=\color{black},
  sensitive=false,
  comment=[l]{//},
  morecomment=[s]{/*}{*/},
  commentstyle=\color{purple}\ttfamily,
  stringstyle=\color{blue}\ttfamily,
  morestring=[b]',
  morestring=[b]"
}

\lstset{
  language=JavaScript,
  backgroundcolor=\color{lightgray},
  extendedchars=true,
  basicstyle=\footnotesize\ttfamily,
  showstringspaces=false,
  showspaces=false,
  numbers=left,
  numberstyle=\footnotesize,
  numbersep=9pt,
  tabsize=2,
  breaklines=true,
  showtabs=false,
}

% Java code with lstlisting
\definecolor{dkgreen}{rgb}{0,0.6,0}
\definecolor{gray}{rgb}{0.5,0.5,0.5}
\definecolor{mauve}{rgb}{0.58,0,0.82}

\lstset{frame=tb,
  language=Java,
  aboveskip=3mm,
  belowskip=3mm,
  showstringspaces=false,
  columns=flexible,
  basicstyle={\small\ttfamily},
  numbers=none,
  numberstyle=\tiny\color{gray},
  keywordstyle=\color{blue},
  commentstyle=\color{dkgreen},
  stringstyle=\color{mauve},
  breaklines=true,
  breakatwhitespace=true,
  tabsize=3
}
\begin{document}
\setlength\parindent{0pt}
% Tikz general settings
\tikzstyle{relation} = [diamond, draw, fill=blue!20, text width=4em,
  text badly centered, node distance=3cm, inner sep=0pt]
\tikzstyle{attribute} = [draw, ellipse, fill=red!20, node distance=2.5cm,
  minimum height=2em]
\tikzstyle{entity} = [rectangle, draw, fill=blue!20, text width=5em,
  text centered, minimum height=4em]
\tikzstyle{relation-weak} = [diamond, double, draw, fill=blue!20, text width=4em,
  text badly centered, node distance=3cm, inner sep=0pt]
\tikzstyle{entity-weak} = [rectangle, draw, double, fill=blue!20, text width=5em,
  text centered, minimum height=4em]
\tikzstyle{line} = [draw, -]
\tikzstyle{arrow} = [draw, -latex', thick]
\tikzstyle{arrow-round} = [draw, -), thick]
\maketitle
\section{JDBC}
JDBC is a Java interface for interacting with a database. It tries to abstract
away differences between various databases.
See the docs \href{http://docs.oracle.com/javase/tutorial/jdbc/basics/}{HERE}.

\subsection{Driver Manager}
\textbf{java.sql.DriverManager}\\
Used to load libraries needed to access database. You can get the Postgres JDBC
driver \href{http://jdbc.postgresql.org/download.html}{HERE}
\begin{lstlisting}[caption=loading the Postgres driver]
  Class.forName("org.postgresql.Driver")
\end{lstlisting}
use \textbf{java -classpath} to add the driver to Java's classpath

\begin{lstlisting}[caption=connecting to a Postgres DB]
  try {
    connection = DriverManager.getConnection(
      "jdbc:postgresql://5432/my_database_name", "username", "password"
    );
  } catch ...
\end{lstlisting}
\subsection{PreparedStatement}
A regular statement builds a query, sends it to the database. The database
parses the statement and turns it into a query plan.
Normal flow is
Query $\rightarrow$ Parser $\rightarrow$ Rewriter $\rightarrow$ Planner\\
A prepared statement is already partially formed - which is better for
performance.
\begin{lstlisting}[caption=prepared statement]
  try {
    PreparedStatement ps =
      // prepare a query from a string containing SQL
      // the ? is a parameter
      String myQuery = "SELECT * FROM NCAA WHERE losers = ?";
      connection.prepareStatement(myQuery);
      ps.setString("the parameter input"); // adds quotes automatically
      ResultSet rs = ps.executeQuery();
      // The ResultSet has a 'cursor' that you can move through the results
      // use next() to iterate over the query results
      // the cursor could also be used to modify the Table
      while (rs.next() {
        //indexed from 1
        String result = rs.getDate(1);
  } catch ...
\end{lstlisting}

\subsubsection{SQL Injection}
\begin{lstlisting}[language=sql,caption=a simple query]
  SELECT *
  FROM Table
  WHERE User = $_GET['user'];
\end{lstlisting}
Suppose somebody inputs \lstinline{"username; DROP DATABASE"}. By adding SQL
statements to their username input, users could run commands on your database.
Prepared statements can protect against this, as it is more difficult to insert
malicious statements into a partially formed query that is only expecting input
of a specific typerun commands on your database. Prepared statements can protect
against this, as it is more difficult to insert malicious statements into a
partially formed query that is only expecting input of a specific type.
\subsubsection{Prepared Statements in Postgres}
You can create prepared statements directly in Postgres.
See \href{http://www.postgresql.org/docs/9.2/static/sql-prepare.html}{PREPARE}

\section{EXPLAIN and ANALYZE}
Used to check performance of SQL statements.
See \href{http://www.postgresql.org/docs/9.2/static/sql-explain.html}{EXPLAIN}
\begin{lstlisting}[language=sql,caption=a simple query]
  EXPLAIN SELECT ... FROM ... WHERE ...;
\end{lstlisting}
If the result shows a high count for rows=, look into ways to limit the rows
explored.
\end{document}

\documentclass[12pt]{article}
\usepackage{graphicx}
\usepackage{float}
\usepackage{amsmath}
\usepackage{sidecap}
\usepackage{fullpage}
\usepackage{hyperref}
\usepackage{listings}
\usepackage{latexsym}
\usepackage{color}
\usepackage{tikz}
\usetikzlibrary{shapes,arrows, matrix, positioning, fit}

% Java code with lstlisting
\definecolor{dkgreen}{rgb}{0,0.6,0}
\definecolor{gray}{rgb}{0.5,0.5,0.5}
\definecolor{mauve}{rgb}{0.58,0,0.82}

\lstset{frame=tb,
    language=Java,
    aboveskip=3mm,
    belowskip=3mm,
    showstringspaces=false,
    columns=flexible,
    basicstyle={\small\ttfamily},
    numbers=none,
    numberstyle=\tiny\color{gray},
    keywordstyle=\color{blue},
    commentstyle=\color{dkgreen},
    stringstyle=\color{mauve},
    breaklines=true,
    breakatwhitespace=true
    tabsize=3
}

\title{CIS4301 Notes: Database Design}
\author{Ryan Roden-Corrent}
\date{\today}

\begin{document}
\setlength\parindent{0pt}
% Tikz general settings
\tikzstyle{relation} = [diamond, draw, fill=blue!20, text width=4em,
  text badly centered, node distance=3cm, inner sep=0pt]
\tikzstyle{attribute} = [draw, ellipse, fill=red!20, node distance=2.5cm,
  minimum height=2em]
\tikzstyle{entity} = [rectangle, draw, fill=blue!20, text width=5em,
  text centered, minimum height=4em]
\tikzstyle{relation-weak} = [diamond, double, draw, fill=blue!20, text width=4em,
  text badly centered, node distance=3cm, inner sep=0pt]
\tikzstyle{entity-weak} = [rectangle, draw, double, fill=blue!20, text width=5em,
  text centered, minimum height=4em]
\tikzstyle{line} = [draw, -]
\tikzstyle{arrow} = [draw, -latex', thick]
\tikzstyle{arrow-round} = [draw, -), thick]
\maketitle
\section{SQL Authoriation}
\href{http://infolab.stanford.edu/~ullman/fcdb/aut07/}{Stanford Slides}
\begin{itemize}
  \item {Privileges}
  \item {Grant and Revoke}
  \item {Grant Diagrams}
\end{itemize}

\subsection{Authoriation}
\begin{itemize}
  \item {A file system identifies certain privileges on the objects in manages}
  \item {A file system tracks these privileges for users, groups, and the world}
  \item {SQL priveleges are more detailed - typically shown in grant diagram}
\end{itemize}

\href{http://www.postgresql.org/docs/9.2/static/sql-grant.html}
{See the Postgres Docs on GRANT}

Priveleges can be granted:
\begin{itemize}
  \item {per user}
  \item {per action
      \begin{itemize}
        \item {SELECT}
        \item {INSERT}
        \item {UPDATE}
        \item {DELETE}
        \item {REFERENCES}
        \item {TRIGGERS}
      \end{itemize}
    }
  \item {per column}
\end{itemize}

\subsubsection{Roles}
\begin{itemize}
  \item {Analogous to users in a filesystem}
  \item {Each role has its own set of permissions}
  \item {Permissions can be inherited from roles}
\end{itemize}

\begin{lstlisting}[language=sql,caption=example query,label={lst:ex1}]
  INSERT INTO Beers(name)
  SELECT beer FROM Sells
  WHERE NOT EXISTS
    (SELECT * FROM Beers
    WHERE name = beer);
\end{lstlisting}

Priveleges Required for \autoref{lst:ex1}:
\begin{itemize}
  \item {SELECT on Sells and Beers}
  \item {INSERT on Beers or Beers.name}
\end{itemize}

\subsection{Permissions in pgsql}
\begin{itemize}
  \item {$\backslash du$ to list roles}
  \item {$\backslash dp$ to list permissions per table
    \begin{itemize}
      \item {Letters represent permissions}
      \item {Look the up on Postgres docs}
    \end{itemize}
  }
  \item {Superuser}
\end{itemize}

\subsection{Views as access control}
Suppose there is a relation \textbf{Emps(name,addr,salary)}
\begin{lstlisting}[language=sql,caption=view to control access,label={lst:ex2}]
  CREATE VIEW SafeEmps AS
  SELECT name, addr FROM Emps;
\end{lstlisting}
In \autoref{lst:ex2}, a view is created to allow access to employee name and
address while hiding their salary. This view is \textbf{updateable}.
A complex join can prevent a view from being updateable if it does not preserve
all information from the original tables.

As an example, consider a view that has the same employees but with all of the
vowels removed from their names. There is no longer a mapping from the modified
employee names back to their original names (for example, the names Josh and
Joshe would both map to Jsh in the view).

\subsection{The GRANT Statement}
\begin{lstlisting}[language=sql,caption=grant format,label={lst:ex3}]
  GRANT list_of_priveleges
  ON relation_or_object
  TO list_of_authorization_ids;
\end{lstlisting}
A GRANT option like this could be added to the end of a script to it is run
every time.

\subsection{The REVOKE Statement}
\begin{lstlisting}[language=sql,caption=revoke format,label={lst:ex4}]
  REVKE list_of_priveleges
  ON relation_or_object
  FROM list_of_authorization_ids;
\end{lstlisting}
\subsubsection{REVOKE Options}
\begin{description}
  \item[CASCADE] any grants made by revokee are no longer in force, no matter
    how far privilege passed
  \item[RESTRICT] if the privilege was passed to others, the REVOKE fails, as a
    warning that something else should be done to chase it down
\end{description}
Suppose Rick was granted \textbf{UPDATE} on Wins table. Rick grants this
privilege to Billy, who grants it to Anthony. The path of permission passage is
tracked in the database. \textbf{REVOKING} this privilege from Billy with the
\textbf{CASCADE} option would automatically revoke Anthony's privilege too,
whereas the \textbf{RESTRICT} option would cause the operation to fail until
Anthony's privilege was revoked too.

However, if Anthony were also granted this privilege by Rick, then he can keep
the permission, as there is still a user who granted him this permission and
still has it themself.

\end{document}
